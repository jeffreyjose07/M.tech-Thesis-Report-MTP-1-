% Chapter Template

\chapter{Introduction} % Main chapter title

\label{Chapter 1} % Change X to a consecutive number; for referencing this chapter elsewhere, use \ref{ChapterX}

\lhead{Chapter 1. \emph{Introduction}} % Change X to a consecutive number; this is for the header on each page - perhaps a shortened title

%----------------------------------------------------------------------------------------
%	SECTION 1
%---------------------------------------------------------------------------------------
Synthetic aperture radar (SAR) can provide more useful information of targets than the optical data from traditional remote sensing. Therefore, the SAR data has been used for various remote sensing applications such as land cover classification, urban extraction, and analysis. Remote sensing has provided invaluable information to forestry. Remote sensing images can be used to map the spatial extent of forest resources. We are interested in the detection of forest cover in an area when provided with its corresponding SAR data.


\section{Motivation}

Forests are a very important natural resource and provides habitats for animals and livelihoods for humans. Forests also offer watershed protection, prevent soil erosion and mitigate climate change. Reliable and timely information on forest area and land use nearby forest is of great importance to planners and policymakers for efficient wildlife and forest conservation and for taking decisions on many other related issues. Optical data have been used for the assessment of mapping forest cover at medium and high resolutions. Mapping of forests usually involves figuring out whether an area is forest cover or not from looking at the image. Nonetheless, the utility of optical data is highly affected by atmospheric conditions. SAR provides an alternative to obtain information. Along with single polarization datasets, dual or fully polarimetric SAR images have been used. SAR remote sensing allows all weather, global scale imaging and estimation of important bio and geophysical parameters about the Earth's surface. It is achieved by sensing scattered electromagnetic fields reflected from the Earth surface when emitted by an electromagnetic energy source situated on an aircraft, spacecraft or satellite outside of the Earth's atmosphere\cite{pottier2009overview}.  The Indian RISAT-1 satellite has provided hybrid polarimetric high resolution data and the data availability has allowed us to evaluate the suitability of hybrid polarimetric data for forest detection and classification\cite{kothapalli2017multi}. 

\section{Problem Statement and Objective}

The general problem this research addresses is defined below:

\textit{Given hybrid polarized SAR data of a region, detect the areas within the region that are forest cover.}

The major objectives of this research are summarized as follows:
\begin{itemize}
\item Classify all pixels in the image under study as either forest or non-forest.  
\item Reduce the human effort required to classify the images.  
\end{itemize}



